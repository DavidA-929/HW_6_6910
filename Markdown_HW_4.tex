\documentclass[12pt,]{article}
\usepackage{lmodern}
\usepackage{amssymb,amsmath}
\usepackage{ifxetex,ifluatex}
\usepackage{fixltx2e} % provides \textsubscript
\ifnum 0\ifxetex 1\fi\ifluatex 1\fi=0 % if pdftex
  \usepackage[T1]{fontenc}
  \usepackage[utf8]{inputenc}
\else % if luatex or xelatex
  \ifxetex
    \usepackage{mathspec}
  \else
    \usepackage{fontspec}
  \fi
  \defaultfontfeatures{Ligatures=TeX,Scale=MatchLowercase}
\fi
% use upquote if available, for straight quotes in verbatim environments
\IfFileExists{upquote.sty}{\usepackage{upquote}}{}
% use microtype if available
\IfFileExists{microtype.sty}{%
\usepackage{microtype}
\UseMicrotypeSet[protrusion]{basicmath} % disable protrusion for tt fonts
}{}
\usepackage[margin=1in]{geometry}
\usepackage{hyperref}
\hypersetup{unicode=true,
            pdftitle={STAT 6910: HW 4},
            pdfauthor={David Angeles},
            pdfborder={0 0 0},
            breaklinks=true}
\urlstyle{same}  % don't use monospace font for urls
\usepackage{color}
\usepackage{fancyvrb}
\newcommand{\VerbBar}{|}
\newcommand{\VERB}{\Verb[commandchars=\\\{\}]}
\DefineVerbatimEnvironment{Highlighting}{Verbatim}{commandchars=\\\{\}}
% Add ',fontsize=\small' for more characters per line
\usepackage{framed}
\definecolor{shadecolor}{RGB}{248,248,248}
\newenvironment{Shaded}{\begin{snugshade}}{\end{snugshade}}
\newcommand{\KeywordTok}[1]{\textcolor[rgb]{0.13,0.29,0.53}{\textbf{#1}}}
\newcommand{\DataTypeTok}[1]{\textcolor[rgb]{0.13,0.29,0.53}{#1}}
\newcommand{\DecValTok}[1]{\textcolor[rgb]{0.00,0.00,0.81}{#1}}
\newcommand{\BaseNTok}[1]{\textcolor[rgb]{0.00,0.00,0.81}{#1}}
\newcommand{\FloatTok}[1]{\textcolor[rgb]{0.00,0.00,0.81}{#1}}
\newcommand{\ConstantTok}[1]{\textcolor[rgb]{0.00,0.00,0.00}{#1}}
\newcommand{\CharTok}[1]{\textcolor[rgb]{0.31,0.60,0.02}{#1}}
\newcommand{\SpecialCharTok}[1]{\textcolor[rgb]{0.00,0.00,0.00}{#1}}
\newcommand{\StringTok}[1]{\textcolor[rgb]{0.31,0.60,0.02}{#1}}
\newcommand{\VerbatimStringTok}[1]{\textcolor[rgb]{0.31,0.60,0.02}{#1}}
\newcommand{\SpecialStringTok}[1]{\textcolor[rgb]{0.31,0.60,0.02}{#1}}
\newcommand{\ImportTok}[1]{#1}
\newcommand{\CommentTok}[1]{\textcolor[rgb]{0.56,0.35,0.01}{\textit{#1}}}
\newcommand{\DocumentationTok}[1]{\textcolor[rgb]{0.56,0.35,0.01}{\textbf{\textit{#1}}}}
\newcommand{\AnnotationTok}[1]{\textcolor[rgb]{0.56,0.35,0.01}{\textbf{\textit{#1}}}}
\newcommand{\CommentVarTok}[1]{\textcolor[rgb]{0.56,0.35,0.01}{\textbf{\textit{#1}}}}
\newcommand{\OtherTok}[1]{\textcolor[rgb]{0.56,0.35,0.01}{#1}}
\newcommand{\FunctionTok}[1]{\textcolor[rgb]{0.00,0.00,0.00}{#1}}
\newcommand{\VariableTok}[1]{\textcolor[rgb]{0.00,0.00,0.00}{#1}}
\newcommand{\ControlFlowTok}[1]{\textcolor[rgb]{0.13,0.29,0.53}{\textbf{#1}}}
\newcommand{\OperatorTok}[1]{\textcolor[rgb]{0.81,0.36,0.00}{\textbf{#1}}}
\newcommand{\BuiltInTok}[1]{#1}
\newcommand{\ExtensionTok}[1]{#1}
\newcommand{\PreprocessorTok}[1]{\textcolor[rgb]{0.56,0.35,0.01}{\textit{#1}}}
\newcommand{\AttributeTok}[1]{\textcolor[rgb]{0.77,0.63,0.00}{#1}}
\newcommand{\RegionMarkerTok}[1]{#1}
\newcommand{\InformationTok}[1]{\textcolor[rgb]{0.56,0.35,0.01}{\textbf{\textit{#1}}}}
\newcommand{\WarningTok}[1]{\textcolor[rgb]{0.56,0.35,0.01}{\textbf{\textit{#1}}}}
\newcommand{\AlertTok}[1]{\textcolor[rgb]{0.94,0.16,0.16}{#1}}
\newcommand{\ErrorTok}[1]{\textcolor[rgb]{0.64,0.00,0.00}{\textbf{#1}}}
\newcommand{\NormalTok}[1]{#1}
\usepackage{graphicx,grffile}
\makeatletter
\def\maxwidth{\ifdim\Gin@nat@width>\linewidth\linewidth\else\Gin@nat@width\fi}
\def\maxheight{\ifdim\Gin@nat@height>\textheight\textheight\else\Gin@nat@height\fi}
\makeatother
% Scale images if necessary, so that they will not overflow the page
% margins by default, and it is still possible to overwrite the defaults
% using explicit options in \includegraphics[width, height, ...]{}
\setkeys{Gin}{width=\maxwidth,height=\maxheight,keepaspectratio}
\IfFileExists{parskip.sty}{%
\usepackage{parskip}
}{% else
\setlength{\parindent}{0pt}
\setlength{\parskip}{6pt plus 2pt minus 1pt}
}
\setlength{\emergencystretch}{3em}  % prevent overfull lines
\providecommand{\tightlist}{%
  \setlength{\itemsep}{0pt}\setlength{\parskip}{0pt}}
\setcounter{secnumdepth}{0}
% Redefines (sub)paragraphs to behave more like sections
\ifx\paragraph\undefined\else
\let\oldparagraph\paragraph
\renewcommand{\paragraph}[1]{\oldparagraph{#1}\mbox{}}
\fi
\ifx\subparagraph\undefined\else
\let\oldsubparagraph\subparagraph
\renewcommand{\subparagraph}[1]{\oldsubparagraph{#1}\mbox{}}
\fi

%%% Use protect on footnotes to avoid problems with footnotes in titles
\let\rmarkdownfootnote\footnote%
\def\footnote{\protect\rmarkdownfootnote}

%%% Change title format to be more compact
\usepackage{titling}

% Create subtitle command for use in maketitle
\newcommand{\subtitle}[1]{
  \posttitle{
    \begin{center}\large#1\end{center}
    }
}

\setlength{\droptitle}{-2em}

  \title{STAT 6910: HW 4}
    \pretitle{\vspace{\droptitle}\centering\huge}
  \posttitle{\par}
    \author{David Angeles}
    \preauthor{\centering\large\emph}
  \postauthor{\par}
    \date{}
    \predate{}\postdate{}
  

\begin{document}
\maketitle

\begin{verbatim}
## Warning: package 'ggplot2' was built under R version 3.4.4
\end{verbatim}

\begin{verbatim}
## Warning: package 'emmeans' was built under R version 3.4.4
\end{verbatim}

\begin{verbatim}
## NOTE: As of emmeans versions > 1.2.3,
##       The 'cld' function will be deprecated in favor of 'CLD'.
##       You may use 'cld' only if you have package:multcomp attached.
\end{verbatim}

\subsection{Problem 1}\label{problem-1}

The meat cooking experiment was described in Exercise 14 of Chap. 3, and
the data were given in Table 3.14, p.~68.

\begin{enumerate}
\def\labelenumi{(\alph{enumi})}
\tightlist
\item
  Compare the effects of the six treatments, pairwise, using Scheffe's
  method of multiple comparisons and a 95\% overall confidence level.
\end{enumerate}

The data displayed below is represented by the codes 1, 2, 3, 4, 5, and
6 which denote the frying fat content at 10\%, 15\%, and 20\% and the
grilling fat content at 10\%, 15\% and 20\% respectively for the
post-cooking weight data (in grams) for the meat cooking experiment.

\begin{Shaded}
\begin{Highlighting}[]
\NormalTok{Post_grams }
\end{Highlighting}
\end{Shaded}

\begin{verbatim}
##       [,1] [,2]
##  [1,]   81    1
##  [2,]   88    1
##  [3,]   85    1
##  [4,]   84    1
##  [5,]   84    1
##  [6,]   85    2
##  [7,]   80    2
##  [8,]   82    2
##  [9,]   80    2
## [10,]   82    2
## [11,]   71    3
## [12,]   77    3
## [13,]   72    3
## [14,]   80    3
## [15,]   80    3
## [16,]   84    4
## [17,]   84    4
## [18,]   82    4
## [19,]   81    4
## [20,]   86    4
## [21,]   83    5
## [22,]   88    5
## [23,]   85    5
## [24,]   86    5
## [25,]   88    5
## [26,]   78    6
## [27,]   75    6
## [28,]   78    6
## [29,]   79    6
## [30,]   82    6
\end{verbatim}

\begin{Shaded}
\begin{Highlighting}[]
\KeywordTok{colnames}\NormalTok{(Post_grams) <-}\StringTok{ }\KeywordTok{c}\NormalTok{( }\StringTok{"Weight"}\NormalTok{, }\StringTok{"Code"}\NormalTok{)}
\NormalTok{Post_grams <-}\StringTok{ }\KeywordTok{data.frame}\NormalTok{(Post_grams)}
\NormalTok{Post_grams}\OperatorTok{$}\NormalTok{Code <-}\StringTok{ }\KeywordTok{factor}\NormalTok{(Post_grams}\OperatorTok{$}\NormalTok{Code)}
\KeywordTok{summary}\NormalTok{(Post_grams)}
\end{Highlighting}
\end{Shaded}

\begin{verbatim}
##      Weight      Code 
##  Min.   :71.00   1:5  
##  1st Qu.:80.00   2:5  
##  Median :82.00   3:5  
##  Mean   :81.67   4:5  
##  3rd Qu.:84.75   5:5  
##  Max.   :88.00   6:5
\end{verbatim}

Let \(\tau_1,\tau_2,...,\tau_6\) represent the true mean of the codes
\(1,2,..,.6\) respectively. We can see that the contrasts have the
following 95\% overall confidence intervals respectively using Scheffe's
method of multiple comparisons and a 95\% overall confidence level.

\begin{Shaded}
\begin{Highlighting}[]
\NormalTok{Post_weight_model <-}\StringTok{ }\KeywordTok{aov}\NormalTok{(Weight }\OperatorTok{~}\StringTok{ }\NormalTok{Code , }\DataTypeTok{data =}\NormalTok{ Post_grams)}
\NormalTok{weight_means <-}\StringTok{ }\KeywordTok{emmeans}\NormalTok{(}\DataTypeTok{object =}\NormalTok{ Post_weight_model, }\DataTypeTok{specs =} \StringTok{"Code"}\NormalTok{)}
\NormalTok{estimates <-}\StringTok{ }\KeywordTok{contrast}\NormalTok{(weight_means,}\DataTypeTok{method =} \StringTok{"pairwise"}\NormalTok{, }\DataTypeTok{adjust =} \StringTok{"bonferroni"}\NormalTok{)}
\NormalTok{alpha <-}\StringTok{ }\FloatTok{0.05}
\NormalTok{v =}\StringTok{ }\DecValTok{6}
\NormalTok{n =}\StringTok{ }\DecValTok{30}
\NormalTok{wS <-}\StringTok{ }\KeywordTok{sqrt}\NormalTok{((v}\OperatorTok{-}\DecValTok{1}\NormalTok{) }\OperatorTok{*}\StringTok{ }\KeywordTok{qf}\NormalTok{(}\DecValTok{1}\OperatorTok{-}\NormalTok{alpha,v}\OperatorTok{-}\DecValTok{1}\NormalTok{,n}\OperatorTok{-}\NormalTok{v))}
\KeywordTok{cbind}\NormalTok{(}\KeywordTok{summary}\NormalTok{(estimates)[,}\StringTok{"estimate"}\NormalTok{] }\OperatorTok{-}\StringTok{ }\NormalTok{wS}\OperatorTok{*}\StringTok{ }\KeywordTok{summary}\NormalTok{(estimates)[,}\StringTok{"SE"}\NormalTok{],}
      \KeywordTok{summary}\NormalTok{(estimates)[,}\StringTok{"estimate"}\NormalTok{] }\OperatorTok{+}\StringTok{ }\NormalTok{wS}\OperatorTok{*}\StringTok{ }\KeywordTok{summary}\NormalTok{(estimates)[,}\StringTok{"SE"}\NormalTok{])}
\end{Highlighting}
\end{Shaded}

\begin{verbatim}
##              [,1]      [,2]
##  [1,]  -3.5714537  8.771454
##  [2,]   2.2285463 14.571454
##  [3,]  -5.1714537  7.171454
##  [4,]  -7.7714537  4.571454
##  [5,]  -0.1714537 12.171454
##  [6,]  -0.3714537 11.971454
##  [7,]  -7.7714537  4.571454
##  [8,] -10.3714537  1.971454
##  [9,]  -2.7714537  9.571454
## [10,] -13.5714537 -1.228546
## [11,] -16.1714537 -3.828546
## [12,]  -8.5714537  3.771454
## [13,]  -8.7714537  3.571454
## [14,]  -1.1714537 11.171454
## [15,]   1.4285463 13.771454
\end{verbatim}

We can observe that the contrasts
\(\tau_1 - \tau_3, \tau_3 - \tau_4, \tau_3 - \tau_5,\) and
\(\tau_5 - \tau_6\) don't include 0 and therefore can conclude that
\(\tau_1 \neq \tau_3, \tau_3 \neq \tau_4, \tau_3 \neq \tau_5,\) and
\(\tau_5 \neq \tau_6\) at the overall \(\alpha=.05\) level.

\subsection{Problem 2}\label{problem-2}

Exercise 15 of Chap. 3 (p.~67) concerns a study of the effects of four
levels of sulfamerazine (0, 5, 10, 15 g per 100 lb of fish) on the
hemoglobin content of trout blood. An analysis of variance test rejected
the hypothesis that the four treatment effects are the same at
significance level \(\alpha=0.01\).

\begin{enumerate}
\def\labelenumi{(\alph{enumi})}
\tightlist
\item
  Compare the four treatments using Tukey's method of pairwise
  comparisons and a 99\% overall confidence level.
\end{enumerate}

Let \(\tau_1,\tau_2,\tau_3,\) and \(\tau_6\) represent the true mean of
the effects of four levels of sulfamerazine 0, 5, 10, and 15 g per 100
lb of fish on the hemoglobin content of trout blood, respectively.

The Tukey's method of pairwise comparisons and 99\% overall confidence
level are displayed below.

\begin{Shaded}
\begin{Highlighting}[]
\NormalTok{Hemoglobin_}\DecValTok{2}\NormalTok{ <-}\StringTok{ }\KeywordTok{matrix}\NormalTok{(}\KeywordTok{c}\NormalTok{( }\KeywordTok{rep}\NormalTok{(}\DecValTok{1}\NormalTok{, }\DataTypeTok{each =} \DecValTok{10}\NormalTok{), }\KeywordTok{rep}\NormalTok{(}\DecValTok{2}\NormalTok{, }\DataTypeTok{each =} \DecValTok{10}\NormalTok{), }
                        \KeywordTok{rep}\NormalTok{(}\DecValTok{3}\NormalTok{, }\DataTypeTok{each =} \DecValTok{10}\NormalTok{), }\KeywordTok{rep}\NormalTok{(}\DecValTok{4}\NormalTok{, }\DataTypeTok{each =} \DecValTok{10}\NormalTok{),}
          \FloatTok{6.7}\NormalTok{, }\FloatTok{7.8}\NormalTok{, }\FloatTok{5.5}\NormalTok{, }\FloatTok{8.4}\NormalTok{ ,}\FloatTok{7.0}\NormalTok{, }\FloatTok{7.8}\NormalTok{, }\FloatTok{8.6}\NormalTok{,}\FloatTok{7.4}\NormalTok{ ,}\FloatTok{5.8}\NormalTok{, }\FloatTok{7.0}\NormalTok{, }
          \FloatTok{9.9}\NormalTok{, }\FloatTok{8.4}\NormalTok{ ,}\FloatTok{10.4}\NormalTok{, }\FloatTok{9.3}\NormalTok{ ,}\FloatTok{10.7}\NormalTok{ ,}\FloatTok{11.9}\NormalTok{, }\FloatTok{7.1}\NormalTok{, }\FloatTok{6.4}\NormalTok{ ,}\FloatTok{8.6}\NormalTok{, }\FloatTok{10.6}\NormalTok{, }
          \FloatTok{10.4}\NormalTok{,}\FloatTok{8.1}\NormalTok{ ,}\FloatTok{10.6}\NormalTok{, }\FloatTok{8.7}\NormalTok{, }\FloatTok{10.7}\NormalTok{, }\FloatTok{9.1}\NormalTok{, }\FloatTok{8.8}\NormalTok{,}\FloatTok{8.1}\NormalTok{, }\FloatTok{7.8}\NormalTok{, }\FloatTok{8.0}\NormalTok{, }
          \FloatTok{9.3}\NormalTok{, }\FloatTok{9.3}\NormalTok{ ,}\FloatTok{7.2}\NormalTok{, }\FloatTok{7.8}\NormalTok{, }\FloatTok{9.3}\NormalTok{, }\FloatTok{10.2}\NormalTok{, }\FloatTok{8.7}\NormalTok{, }\FloatTok{8.6}\NormalTok{, }\FloatTok{9.3}\NormalTok{, }\FloatTok{7.2}\NormalTok{), }
          \DataTypeTok{byrow =} \OtherTok{FALSE}\NormalTok{, }\DataTypeTok{ncol =} \DecValTok{2}\NormalTok{ )}
\NormalTok{Hemoglobin_}\DecValTok{2}\NormalTok{ <-}\StringTok{ }\KeywordTok{data.frame}\NormalTok{(Hemoglobin_}\DecValTok{2}\NormalTok{)}
\NormalTok{Hemoglobin_}\DecValTok{2}\OperatorTok{$}\NormalTok{X1 <-}\StringTok{ }\KeywordTok{as.factor}\NormalTok{(Hemoglobin_}\DecValTok{2}\OperatorTok{$}\NormalTok{X1)}
\KeywordTok{colnames}\NormalTok{(Hemoglobin_}\DecValTok{2}\NormalTok{) <-}\StringTok{ }\KeywordTok{c}\NormalTok{(}\StringTok{"Codes"}\NormalTok{, }\StringTok{"Hemo"}\NormalTok{)}

\NormalTok{hem<-}\StringTok{ }\KeywordTok{aov}\NormalTok{(Hemo }\OperatorTok{~}\StringTok{ }\NormalTok{Codes, }\DataTypeTok{data =}\NormalTok{ Hemoglobin_}\DecValTok{2}\NormalTok{)}
\NormalTok{hemoglobin_means<-}\StringTok{ }\KeywordTok{emmeans}\NormalTok{(}\DataTypeTok{object =}\NormalTok{ hem, }\DataTypeTok{specs =} \StringTok{"Codes"}\NormalTok{)}

\NormalTok{hemo.tukey <-}\StringTok{ }\KeywordTok{contrast}\NormalTok{(hemoglobin_means,}\DataTypeTok{method =} \StringTok{"pairwise"}\NormalTok{, }\DataTypeTok{adjust =} \StringTok{"tukey"}\NormalTok{)}
\KeywordTok{summary}\NormalTok{(hemo.tukey,}\DataTypeTok{infer =} \KeywordTok{c}\NormalTok{(}\OtherTok{TRUE}\NormalTok{,}\OtherTok{TRUE}\NormalTok{), }\DataTypeTok{level =}\NormalTok{ .}\DecValTok{99}\NormalTok{)}
\end{Highlighting}
\end{Shaded}

\begin{verbatim}
##  contrast estimate        SE df  lower.CL    upper.CL t.ratio p.value
##  1 - 2       -2.13 0.5601141 36 -4.003146 -0.25685355  -3.803  0.0029
##  1 - 3       -1.83 0.5601141 36 -3.703146  0.04314645  -3.267  0.0122
##  1 - 4       -1.49 0.5601141 36 -3.363146  0.38314645  -2.660  0.0539
##  2 - 3        0.30 0.5601141 36 -1.573146  2.17314645   0.536  0.9498
##  2 - 4        0.64 0.5601141 36 -1.233146  2.51314645   1.143  0.6660
##  3 - 4        0.34 0.5601141 36 -1.533146  2.21314645   0.607  0.9292
## 
## Confidence level used: 0.99 
## Conf-level adjustment: tukey method for comparing a family of 4 estimates 
## P value adjustment: tukey method for comparing a family of 4 estimates
\end{verbatim}

We can see from the data that the only contrast that has a \(p\)-value
less that \(0.01\) is \(\tau_1-\tau_2\). So we can conclude that
\(\tau_1 \neq \tau_2\) at the \(\alpha=0.1\) level.

\begin{enumerate}
\def\labelenumi{(\alph{enumi})}
\setcounter{enumi}{1}
\tightlist
\item
  Compare the effect of no sulfamerazine on the hemoglobin content of
  trout blood with the average effect of the other three levels. The
  overall confidence level of all intervals in parts (a) and (b) should
  be at least 98\%.
\end{enumerate}

We will use an \(\alpha = .01\) level in order to keep the overall
confidence level of all intervals at least 98\%. From the code below, we
obtain a \(p\)-value of \(p=0.03\) for the contract \((3,-1,-1,-1)\)
which is greater than \(\alpha=.01\). So our data do not provide
evidence that no sulfamerazine on the hemoglobin content differs from
the average effect of the other three levels.

\begin{Shaded}
\begin{Highlighting}[]
\KeywordTok{contrast}\NormalTok{(hemoglobin_means,}\DataTypeTok{method =} \KeywordTok{list}\NormalTok{(}\KeywordTok{c}\NormalTok{(}\DecValTok{3}\NormalTok{,}\OperatorTok{-}\DecValTok{1}\NormalTok{,}\OperatorTok{-}\DecValTok{1}\NormalTok{,}\OperatorTok{-}\DecValTok{1}\NormalTok{)))}
\end{Highlighting}
\end{Shaded}

\begin{verbatim}
##  contrast         estimate       SE df t.ratio p.value
##  c(3, -1, -1, -1)    -5.45 1.371994 36  -3.972  0.0003
\end{verbatim}

\subsection{Problem 3}\label{problem-3}

The soap experiment was described in Sect. 2.5.1, p.~20, and an analysis
was given in Sect. 3.7.2, p.~50.

\begin{enumerate}
\def\labelenumi{(\alph{enumi})}
\tightlist
\item
  Suppose that the experimenter had been interested only in the contrast
  \(\tau_1 - \frac{1}{2}(\tau_2 + \tau_3)\), which compares the weight
  loss for the regular soap with the average weight loss for the other
  two soaps. Calculate a confidence interval for this single contrast.
\end{enumerate}

Since it is being assumed that the experimenter might have been
interested only in one contrast, there isn't a need to use a family-wise
error rate. Therefore, using an \(\alpha\)-level of \(\alpha=.05\) and
the results on pg. 51, we can calculate a confidence interval for this
single contrast as follow:

\[
\begin{aligned}
& \sum_{i=1}^v b_i\hat{Y}_{i\bullet} \pm t_{n-v,\alpha/2} \sqrt{MSE \sum_{i=1}^v \frac{b_i^2}{r_i}}\\
&= -0.035- \frac{1}{2}(2.7) - \frac{1}{2}(1.9925) \pm t_{9,.025}
 \sqrt{0.0772 \sum_{i=1}^v \frac{b_i^2}{r_i}}\\
&= -2.38125 \pm 2.262\sqrt{0.0772 \left( \frac{1}{4} + \frac{1/4}{4} + \frac{1/4}{4} \right)}\\
&= -2.38125 \pm 2.262\sqrt{0.0772 \left( \frac{3}{8}  \right)}\\
&= -2.38125 \pm .38487
\end{aligned}
\] Therefore the 95\% confidence interval for the specific contrast is
(-2.76612,-1.99638).

\begin{enumerate}
\def\labelenumi{(\alph{enumi})}
\setcounter{enumi}{1}
\tightlist
\item
  Test the hypothesis that the regular soap has the same average weight
  loss as the average of the other two soaps. Do this via your
  confidence interval in part (a) and also via (4.3.13) and (4.3.15).
\end{enumerate}

For this test, we will use the following:

\begin{center}
 The Null Hypothesis of No Difference Between Regular Soap and Others  $H_0 : \tau_1 = \frac{1}{2}(\tau_2+ \tau_4)$ \\
 vs. The Alternative Hypothesis of Some Difference Between Regular Soap and Others $H_A : \tau_1 \neq \frac{1}{2}(\tau_2+ \tau_4)$
\end{center}

Using the confidence interval, we see that 0 is not in the interval and
therefore we can reject the null hypothesis at the \(\alpha=.05\) level
in favor of the alternative hypothesis that the regular soap does not
have the same average weight loss as the average of the other two soaps.
Next, using Equation 4.3.13 we get

\[
 \begin{aligned}
 \Bigg| \frac{\sum c_i \bar{y}_i}{\sqrt{msE \sum c_i^2/r_i}}  \Bigg|
 &=  \Bigg| \frac{-2.38125}{\sqrt{0.0772 \left( \frac{3}{8} \right)}}  \Bigg| = 13.99525 > 2.262 = t_{9,.025}.
  \end{aligned}
 \] So we can reject the null hypothesis again. Lastly, Using Equation
4.3.15 we get

\[
 \begin{aligned}
 \Bigg| \frac{ \frac{\left(\sum c_i \bar{y}_i\right)^2}{\sum c_i^2/r_i} }{ msE }  \Bigg|
 &=   \Bigg| \frac{ \frac{5.67035 }{3/8} }{ .0772 }  \Bigg|\\
 &=195.8671 > 5.12 = F_{1,9,.05}.
  \end{aligned}
 \] and therefore we can reject the null hypothesis in favor of the
alternative hypothesis again.

\subsection{Problem 4}\label{problem-4}

Consider again the trout experiment in Exercise 15 of Chap. 3.

\begin{enumerate}
\def\labelenumi{(\alph{enumi})}
\tightlist
\item
  Suppose the experiment were to be repeated. Suggest the largest likely
  value for the error mean square \(msE\).
\end{enumerate}

From the code below we have that a 95\% upper confidence limit for
\(\sigma^2\) is 2.43. Therefore, the largest likely value for the error
mean square \(msE\) is \(2.43\).

\begin{Shaded}
\begin{Highlighting}[]
\NormalTok{Hemoglobin_model <-}\StringTok{ }\KeywordTok{aov}\NormalTok{(Hemo }\OperatorTok{~}\StringTok{ }\NormalTok{Codes , }\DataTypeTok{data =}\NormalTok{ Hemoglobin_}\DecValTok{2}\NormalTok{)}
\KeywordTok{anova}\NormalTok{(Hemoglobin_model)}
\end{Highlighting}
\end{Shaded}

\begin{verbatim}
## Analysis of Variance Table
## 
## Response: Hemo
##           Df Sum Sq Mean Sq F value   Pr(>F)   
## Codes      3 26.803  8.9343  5.6955 0.002685 **
## Residuals 36 56.471  1.5686                    
## ---
## Signif. codes:  0 '***' 0.001 '**' 0.01 '*' 0.05 '.' 0.1 ' ' 1
\end{verbatim}

\begin{Shaded}
\begin{Highlighting}[]
\NormalTok{ssE <-}\StringTok{ }\KeywordTok{anova}\NormalTok{(Hemoglobin_model)[}\DecValTok{2}\NormalTok{,}\StringTok{"Sum Sq"}\NormalTok{]}
\NormalTok{X_2_star <-}\StringTok{ }\KeywordTok{qchisq}\NormalTok{(}\FloatTok{0.05}\NormalTok{,}\DataTypeTok{df =} \DecValTok{36}\NormalTok{)}
\NormalTok{ssE}\OperatorTok{/}\StringTok{ }\NormalTok{X_2_star}
\end{Highlighting}
\end{Shaded}

\begin{verbatim}
## [1] 2.426918
\end{verbatim}

\begin{enumerate}
\def\labelenumi{(\alph{enumi})}
\setcounter{enumi}{1}
\tightlist
\item
  How many observations should be taken on each treatment so that the
  length of each interval in a set of simultaneous 95\% confidence
  intervals for pairwise comparisons should be at most 2 g per 100 ml?
\end{enumerate}

Assuming that we want to look at all pairwise comparisons, we know that
tukey's method will provide the narrowest confidence interval set of
simultaneous 95\% confidence intervals for pairwise comparisons.
Therefore, using \(q_{4,4r-4,.05}\), I'll assume \(\sigma^2 = 2.43\)
from part (a), and also that \(r_i = r\).

So we want

\[
1 = \text{ interval half-width }= \frac{1}{\sqrt{2}} q_{4,4r-4,.05} \sqrt{\sigma^2 \left( \frac{2}{r}\right)}.
\] By trial and error we have that \(r= 33\) observations should be
taken on each treatment so that the length of each interval in a set of
simultaneous 95\% confidence intervals for pairwise comparisons is at
most 2 g per 100 ml.

\[ \frac{1}{\sqrt{2}} 3.68  \sqrt{2.43 \left( \frac{2}{33}\right)} = .9984\]

\begin{Shaded}
\begin{Highlighting}[]
\KeywordTok{qtukey}\NormalTok{(}\FloatTok{0.95}\NormalTok{,}\DecValTok{4}\NormalTok{,}\DecValTok{128}\NormalTok{)}
\end{Highlighting}
\end{Shaded}

\begin{verbatim}
## [1] 3.681343
\end{verbatim}


\end{document}
